\documentclass[a4paper, 11pt]{article}

% --- IMPOSTAZIONI DI PAGINA (CRUCIALE PER LE 6 PAGINE) ---
% Margini ridotti a 2cm per guadagnare spazio vitale
\usepackage[left=2cm, right=2cm, top=2cm, bottom=2cm]{geometry}

% --- LINGUA E CODIFICA ---
\usepackage[utf8]{inputenc}
\usepackage[T1]{fontenc}
\usepackage[italian]{babel}     % Traduce "Figure", "Table", ecc.
\usepackage{microtype}          % Migliora la spaziatura e la sillabazione (aspetto più professionale)

% --- MATEMATICA E TABELLE ---
\usepackage{amsmath, amssymb}   % Simboli matematici
\usepackage{booktabs}           % Per tabelle professionali (senza linee verticali)
\usepackage{array}

% --- GRAFICI E IMMAGINI ---
\usepackage{graphicx}
\usepackage{subcaption}         % Fondamentale per mettere grafici affiancati (Fig 1a, 1b)
\usepackage{float}              % Per usare [H] e bloccare le immagini dove vuoi tu
\usepackage[colorlinks=true,      % Attiva i colori invece dei riquadri
linkcolor=black,      % Colore link interni (es. Indice, Figure) - meglio black per stampare
urlcolor=blue,        % Colore link web (es. il tuo dataset)
citecolor=black       % Colore citazioni bibliografiche
]{hyperref}% Per link cliccabili (es. fonte dataset) senza rettangoli rossi

% --- CONFIGURAZIONE CODICE R (PER APPENDICE) ---
\usepackage{listings}
\usepackage{xcolor}
\usepackage{lmodern}
\usepackage{enumitem}

\definecolor{codegreen}{rgb}{0,0.6,0}
\definecolor{codegray}{rgb}{0.5,0.5,0.5}
\definecolor{codepurple}{rgb}{0.58,0,0.82}
\definecolor{backcolour}{rgb}{0.95,0.95,0.92}

\lstset{ 
	language=R,
	backgroundcolor=\color{backcolour},   
	commentstyle=\color{codegreen},
	keywordstyle=\color{blue},
	numberstyle=\tiny\color{codegray},
	stringstyle=\color{codepurple},
	basicstyle=\footnotesize\ttfamily, % Font piccolo monospaziato
	breakatwhitespace=false,         
	breaklines=true,                 % Va a capo se la riga è troppo lunga
	captionpos=b,                    
	keepspaces=true,                 
	numbers=left,                    % Numeri di riga a sinistra
	numbersep=5pt,                  
	showspaces=false,                
	showstringspaces=false,
	showtabs=false,                  
	tabsize=2,
	frame=single                     % Cornice attorno al codice
}

% --- INIZIO DEL DOCUMENTO ---
\begin{document}
	
	% --- INTESTAZIONE COMPATTA (Custom Header) ---
	% Non uso \maketitle standard perché occupa mezza pagina inutile.
	\begin{center}
		% Nome del gruppo (evocativo come richiesto)
		{\Large \textbf{GRUPPO: CORES}} \\ 
		\vspace{0.2cm}
		% Titolo del corso e Report
		{\large \textsc{Statistica computazionale -- Report Finale}} \\
		\vspace{0.1cm}
		% Membri del gruppo
		\small
		\begin{tabular}{c c c}
			Maccianti Federico & Rapacioli Nicola & Riva Pietro \\
			(909656) & (915439) & (908813) 
		\end{tabular}
	\end{center}
	
	\hrule
	\vspace{0.5cm}
	
	
	\section{Introduzione}
	Il presente report analizza un dataset ottenuto tramite il repository \href{https://github.com/TracingInsights/2025.git}{TracingInsights}.
	Lo studio si concentra sui dati di telemetria relativi alle sessioni di qualifica della stagione di Formula 1 2025, selezionando per ciascun pilota il singolo giro migliore.
	
	Il dataset originale include le seguenti variabili:
	\begin{center}
		\begin{tabular}{lll}
			\hline
			\textbf{Variabile} & \textbf{Unità} & \textbf{Tipo} \\
			\hline
			Gran premio            & --                 & \texttt{character} \\
			Pilota                 & --                 & \texttt{character} \\
			Tempo dal via          & $\mathrm{s}$        & \texttt{numeric} \\
			Distanza percorsa      & $\mathrm{m}$        & \texttt{numeric} \\
			Distanza relativa      & $0$--$1$            & \texttt{numeric} \\
			Velocità               & $\mathrm{km/h}$     & \texttt{numeric} \\
			Regime motore          & RPM                & \texttt{numeric} \\
			Marcia                 & $1$--$8$            & \texttt{numeric} \\
			Freno                  & $0/1$               & \texttt{factor} \\
			Acceleratore           & $0$--$100\,\%$      & \texttt{numeric} \\
			DRS                    & $0/1$               & \texttt{factor} \\
			Accelerazioni          & $\mathrm{m/s^2}$    & \texttt{numeric} \\
			Coordinate spaziali    & $\mathrm{m}$        & \texttt{numeric} \\
			\hline
		\end{tabular}
	\end{center}
	Le variabili binarie sono codificate come $0$ (non attivo) e $1$ (attivo).
	
	L’obiettivo del presente report è analizzare il dataset descritto, al fine di caratterizzare il comportamento dei piloti durante il giro di gara attraverso le principali variabili telemetriche disponibili.
	
	\section{Analisi Esplorativa}
	\subsection{Considerazioni sulle variabili}
	Poiché lo stile di guida non è riconducibile a variabili di tipo posizionale, le coordinate spaziali e le misure di distanza, sia assolute sia relative, vengono escluse dall’analisi.
	In questa fase preliminare, lo stile di guida viene descritto attraverso variabili dinamiche quali l’utilizzo dell’acceleratore e del freno e le accelerazioni longitudinale e laterale, che consentono di caratterizzare rispettivamente le modalità di decelerazione in ingresso curva e l’intensità con cui la curva viene affrontata.
	In modo tale da poter confrontare tra i vari gran premi le variabili di accelerazione laterale e longitudinale vengono riscalate nell'intervallo $[-1,1]$
	Come si evince dalla Figura \ref{fig:istogrammi_iniziali}, la distribuzione \dots
	
	% Esempio placeholder per due grafici affiancati (risparmio spazio)
	\begin{figure}[ht]
		\centering
		\begin{subfigure}[b]{0.45\textwidth}
			% \includegraphics[width=\textwidth]{tuo_grafico_1.pdf}
			\rule{\textwidth}{3cm} % Rimuovi questa riga e usa includegraphics
			\caption{Distribuzione Variabile X}
		\end{subfigure}
		\hfill
		\begin{subfigure}[b]{0.45\textwidth}
			% \includegraphics[width=\textwidth]{tuo_grafico_2.pdf}
			\rule{\textwidth}{3cm} % Rimuovi questa riga e usa includegraphics
			\caption{Scatterplot X vs Y}
		\end{subfigure}
		\caption{Analisi esplorativa iniziale delle variabili principali.}
		\label{fig:istogrammi_iniziali}
	\end{figure}
	
	\section{Modellazione}
	Abbiamo applicato un modello di regressione lineare:
	\begin{equation}
		Y_i = \beta_0 + \beta_1 X_{1i} + \epsilon_i
	\end{equation}
	
	% --- CHIUSURA DOCUMENTO E APPENDICE ---
	
	% \newpage % Scommenta se vuoi l'appendice in pagina nuova
	\appendix
	\section{Codice R Commentato}
	Di seguito riportiamo lo script utilizzato per l'analisi.
	
	% Se hai il file .R nella stessa cartella:
	% \lstinputlisting[language=R]{analisi_script.R}
	
	% Se vuoi incollare il codice qui:
	\begin{lstlisting}
		# Caricamento librerie
		library(ggplot2)
		data <- read.csv("dataset.csv")
		
		# Analisi preliminare
		summary(data)
	\end{lstlisting}
	
\end{document}