\documentclass[a4paper, 11pt]{article}

% --- IMPOSTAZIONI DI PAGINA (CRUCIALE PER LE 6 PAGINE) ---
% Margini ridotti a 2cm per guadagnare spazio vitale
\usepackage[left=2cm, right=2cm, top=2cm, bottom=2cm]{geometry}

% --- LINGUA E CODIFICA ---
\usepackage[utf8]{inputenc}
\usepackage[T1]{fontenc}
\usepackage[italian]{babel}     % Traduce "Figure", "Table", ecc.
\usepackage{microtype}          % Migliora la spaziatura e la sillabazione (aspetto più professionale)

% --- MATEMATICA E TABELLE ---
\usepackage{amsmath, amssymb}   % Simboli matematici
\usepackage{booktabs}           % Per tabelle professionali (senza linee verticali)
\usepackage{array}

% --- GRAFICI E IMMAGINI ---
\usepackage{graphicx}
\usepackage{subcaption}         % Fondamentale per mettere grafici affiancati (Fig 1a, 1b)
\usepackage{float} 				   % Per usare [H] e bloccare le immagini dove vuoi tu
\usepackage{wrapfig}          
\usepackage[colorlinks=true,      % Attiva i colori invece dei riquadri
linkcolor=black,      % Colore link interni (es. Indice, Figure) - meglio black per stampare
urlcolor=blue,        % Colore link web (es. il tuo dataset)
citecolor=black       % Colore citazioni bibliografiche
]{hyperref}% Per link cliccabili (es. fonte dataset) senza rettangoli rossi

% --- CONFIGURAZIONE CODICE R (PER APPENDICE) ---
\usepackage{listings}
\usepackage{xcolor}
\usepackage{lmodern}
\usepackage{enumitem}

\definecolor{codegreen}{rgb}{0,0.6,0}
\definecolor{codegray}{rgb}{0.5,0.5,0.5}
\definecolor{codepurple}{rgb}{0.58,0,0.82}
\definecolor{backcolour}{rgb}{0.95,0.95,0.92}

\lstset{ 
	language=R,
	backgroundcolor=\color{backcolour},   
	commentstyle=\color{codegreen},
	keywordstyle=\color{blue},
	numberstyle=\tiny\color{codegray},
	stringstyle=\color{codepurple},
	basicstyle=\footnotesize\ttfamily, % Font piccolo monospaziato
	breakatwhitespace=false,         
	breaklines=true,                 % Va a capo se la riga è troppo lunga
	captionpos=b,                    
	keepspaces=true,                 
	numbers=left,                    % Numeri di riga a sinistra
	numbersep=5pt,                  
	showspaces=false,                
	showstringspaces=false,
	showtabs=false,                  
	tabsize=2,
	frame=single                     % Cornice attorno al codice
}

% --- INIZIO DEL DOCUMENTO ---
\begin{document}
	
	% --- INTESTAZIONE COMPATTA (Custom Header) ---
	% Non uso \maketitle standard perché occupa mezza pagina inutile.
	\begin{center}
		% Nome del gruppo (evocativo come richiesto)
		{\Large \textbf{GRUPPO: CORES}} \\ 
		\vspace{0.2cm}
		% Titolo del corso e Report
		{\large \textsc{Statistica computazionale -- Report Finale}} \\
		\vspace{0.1cm}
		% Membri del gruppo
		\small
		\begin{tabular}{c c c}
			Maccianti Federico & Rapacioli Nicola & Riva Pietro \\
			(909656) & (915439) & (908813) 
		\end{tabular}
	\end{center}
	
	\hrule
	\vspace{0.5cm}
	
	
	\section{Introduzione}
	Il presente report analizza un dataset ottenuto tramite il repository \href{https://github.com/TracingInsights/2025.git}{TracingInsights}.
	Lo studio si concentra sui dati di telemetria relativi alle sessioni di qualifica della stagione di Formula 1 2025, selezionando per ciascun pilota il singolo giro migliore.
	
	Il dataset originale include le seguenti variabili:
	\begin{center}
		\begin{tabular}{lll}
			\hline
			\textbf{Variabile} & \textbf{Unità} & \textbf{Tipo} \\
			\hline
			Gran premio            & --                 & \texttt{character} \\
			Pilota                 & --                 & \texttt{character} \\
			Tempo dal via          & $\mathrm{s}$        & \texttt{numeric} \\
			Distanza percorsa      & $\mathrm{m}$        & \texttt{numeric} \\
			Distanza relativa      & $0$--$1$            & \texttt{numeric} \\
			Velocità               & $\mathrm{km/h}$     & \texttt{numeric} \\
			Regime motore          & RPM                & \texttt{numeric} \\
			Marcia                 & $1$--$8$            & \texttt{numeric} \\
			Freno                  & $0/1$               & \texttt{factor} \\
			Acceleratore           & $0$--$100\,\%$      & \texttt{numeric} \\
			DRS                    & $0/1$               & \texttt{factor} \\
			Accelerazioni laterale e longitudinale         & $\mathrm{g}$    & \texttt{numeric} \\
			Coordinate spaziali x,y,z    & $\mathrm{m}$        & \texttt{numeric} \\
			\hline
		\end{tabular}
	\end{center}
	Le variabili binarie sono codificate come $0$ (non attivo) e $1$ (attivo).
	
	L’obiettivo del presente report è analizzare il dataset descritto, al fine di caratterizzare il comportamento dei piloti durante il giro di qualifica attraverso le principali variabili telemetriche disponibili.
	
	\section{Analisi Esplorativa}
	\subsection{Considerazioni sulle variabili}
	Poiché lo stile di guida non è riconducibile a variabili di tipo posizionale, le coordinate spaziali e le misure di distanza, sia assolute sia relative, vengono escluse dall’analisi.
	\begin{wrapfigure}{r}{0.5\textwidth} % {r} sta per destra (right), 0.5 è la larghezza
		\centering
		\includegraphics[width=.63\textwidth]{tel_ex1.pdf}
		\caption{Accelerazioni VER, LEC, NOR.}
		\vspace{-10pt} % Rimuove spazio bianco sotto la didascalia
		\label{fig:confronto}
	\end{wrapfigure}
	
	In questa fase preliminare, lo stile di guida viene descritto attraverso variabili dinamiche quali l’utilizzo dell’acceleratore e del freno e le accelerazioni, longitudinale e laterale, che consentono di caratterizzare rispettivamente le modalità di decelerazione in ingresso curva e l’intensità con cui la curva viene affrontata.
	A sostegno delle ipotesi sopra citate, si riporta la Figura \ref{fig:confronto} che confronta nel Gran Premio degli USA le accelerazioni per i piloti: Charles Leclerc, Lando Norris e Max Verstappen. Notando infatti come nelle variazioni repentine si contraddistinguano meglio i piloti
	\clearpage
	\subsubsection{Trasformazione e creazione di nuove variabili}
	Al fine di rendere confrontabili i diversi tracciati dei Gran Premi, le variabili di accelerazione laterale e longitudinale vengono riscalate, per ciascun Gran Premio e Pilota, nell’intervallo $[-1,1]$. Successivamente, tali variabili vengono trasformate in valore assoluto, così da ottenere una misura della loro intensità complessiva (magnitudo), il cui dominio risulta compreso in $[0,1]$.  
	Per le restanti variabili telemetriche si mantiene invece la forma grezza.
		
	Si creano tre nuove variabili di variazione percentuali avendo notato dalla figura \ref{fig:confronto} che quando ci sono variazioni repentine che si contraddistingue lo stile di guida dei piloti:
	\begin{itemize}[noitemsep, topsep=0pt]
		\item \textbf{\_lag1} : variazione percentuale rispetto all'osservazione precedente
		\item \textbf{\_lag2} : variazione percentuale rispetto a due osservazioni precedenti
		\item \textbf{\_lag3} : variazione percentuale rispetto a tre osservazioni precedenti
	\end{itemize}
	Vengono calcolate indicando con $x_t$ l'unità statistica al tempo $t$ viene calcolata tramite la seguente formula:
	\[\Delta=\dfrac{x_{t}-x_{t-k}}{x_{t-k}}\] 
	
	con $k=1,2,3$ numero di lag per le misure di accelerazione, nel caso in cui $x_{t-k}=0$ si procede con la sostituzione $x_{t-k}=0.01$ per riuscire a mantenere la validità del calcolo e continuità delle dinamiche telemetriche.
	
	Queste variabili consentono di catturare la dinamica delle grandezze nel tempo e sono utili per caratterizzare l’evoluzione del comportamento di guida tra un istante e l’altro.
	
	\subsection{Statistiche riassuntive}
	Si procede al calcolo delle statistiche descrittive al fine di valutare in che modo esse possano essere associate allo stile di guida, considerando separatamente ciascun Gran Premio e Pilota.
	
	Per le misure di accelerazione longitudinale e laterale vengono calcolate media, deviazione standard, valore massimo e valore minimo.
	
	La variabile relativa alla frenata non viene invece inclusa nell’analisi descrittiva, in quanto la sua natura binaria e la forte dipendenza dalle caratteristiche specifiche del singolo Gran Premio rendono poco informative misure sintetiche come media e deviazione standard. Tali indicatori risulterebbero infatti più rappresentativi del tracciato e delle condizioni di gara che dello stile di guida del pilota.
	
	Anche per la variabile accelerazione non vengono considerati gli estremi minimo e massimo, poiché la misura è limitata nell’intervallo [0,100] e tali valori risulterebbero costanti per tutti i piloti e per tutti i Gran Premi, non apportando informazione discriminante.
	
	
	Per le variabili di tipo \_lag$k$ vengono calcolate medie e deviazioni standard distinguendo tra variazioni positive e negative, così da evidenziare eventuali asimmetrie nel comportamento dinamico del pilota.

	
	Nel calcolo delle statistiche descrittive, viene osservato come per il pilota Russell al Gran Premio di Miami si producano degli \texttt{NA}. Ricontrollando i dati grezzi ci si accorge di come probabilmente i sensori telemetrici abbiano avuto un'avaria, in quanto i record sono la maggior risulta pari a zero. Per i motivi elencati sopra, si procede dunque ad eliminare il record.
	
	Il nuovo dataset pertanto contiene 48 variabili, che forniscono informazione sullo stile di guida del pilota nella specifica gara.
	
	Si procede dunque con una analisi delle componenti principali per ridurre la dimensionalità, e comprendere quali siano le variabili più significative nel fornire l'informazione.
	\subsection{Analisi delle componenti principali}
	Per evitare i casi di multicollinearità, avendo correlazioni molto elevate, attraverso la funzione \texttt{findCorrelation} implementata nella libreria : \texttt{\{caret\}}. 
	
	
	
	
	\clearpage
	Come si evince dalla Figura \ref{fig:istogrammi_iniziali}, la distribuzione \dots
	
	% Esempio placeholder per due grafici affiancati (risparmio spazio)
	\begin{figure}[ht]
		\centering
		\begin{subfigure}[b]{0.45\textwidth}
			% \includegraphics[width=\textwidth]{tuo_grafico_1.pdf}
			\rule{\textwidth}{3cm} % Rimuovi questa riga e usa includegraphics
			\caption{Distribuzione Variabile X}
		\end{subfigure}
		\hfill
		\begin{subfigure}[b]{0.45\textwidth}
			% \includegraphics[width=\textwidth]{tuo_grafico_2.pdf}
			\rule{\textwidth}{3cm} % Rimuovi questa riga e usa includegraphics
			\caption{Scatterplot X vs Y}
		\end{subfigure}
		\caption{Analisi esplorativa iniziale delle variabili principali.}
		\label{fig:istogrammi_iniziali}
	\end{figure}
	
	\section{Modellazione}
	Abbiamo applicato un modello di regressione lineare:
	\begin{equation}
		Y_i = \beta_0 + \beta_1 X_{1i} + \epsilon_i
	\end{equation}
	
	% --- CHIUSURA DOCUMENTO E APPENDICE ---
	
	% \newpage % Scommenta se vuoi l'appendice in pagina nuova
	\appendix
	\section{Codice R Commentato}
	Di seguito riportiamo lo script utilizzato per l'analisi.
	
	% Se hai il file .R nella stessa cartella:
	% \lstinputlisting[language=R]{analisi_script.R}
	
	% Se vuoi incollare il codice qui:
	\begin{lstlisting}
		# Caricamento librerie
		library(ggplot2)
		data <- read.csv("dataset.csv")
		
		# Analisi preliminare
		summary(data)
	\end{lstlisting}
	
\end{document}